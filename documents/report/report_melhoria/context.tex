\chapter{Context} \label{ch:context}

\section{Specifications} \label{sec:specs}

\subsection{Technology Overview}

Before beginning analysis and development of the project's application, the
Orange team decided from the various technologies  listed in the specifications
given the necessity of each task that was needed to complete this project.
During this project, the team relied solely on Open Source Technologies as
intended by the specifications. These technologies are not abridged by copyleft
licenses like the Gnu Public License, but some remain GPL-compatible.

In the next sections the technologies used are separated by task context and 
specifications wise.

\subsubsection{Programming languages}

\paragraph{PHP}

This programming language is a standard language for quickly writing web
applications.  It is a dynamically typed, interpreted language with garbage
collection and supports Object Oriented Programming, providing a syntax similar
to older programming languages like C and Perl.

The project is essentially be written in this programming language, given that
is the most easy to setup and learn amongst academics.

\paragraph{Plonk}

For basic application workflow, the application uses the Plonk library - a
subset of the Spoon\footnote{http://www.spoon-library.com/} library - that
automates functionalities like page routing, session handling and page
templating.

\paragraph{Python}

The Python programing language is an interpreted, dynamically typed programming
language with garbage collection, used in the P7-MATCH package for it's use of
the Natural Language Toolkit.

\paragraph{NLTK}

The Natural Language Toolkit is a Python suite of libraries designed for natural
language processing, convenient for matching students' courses and equivalents
between HEI's.

\paragraph{JavaScript}

Javascript is an interpreted, dynamic and weakly typed programming language
mostly used in client side (Web Browser) scripting. It has some Object Oriented
support with garbage collection. This project uses javascript to reduce
server-side load by performing tasks like form validation, generating content,
Ajax3 calls to the server, etc.

\paragraph{jQuery}

jQuery is a pluggable javascript library that speeds up client-side script
development by implementing a Domain Specific Language capable of quickly
processing XHTML DOM elements for various aesthetic purposes. It also provides
convenient functions prepared for Ajax communication or pre-programmed plugins
like calendars, color picker windows, etc. 

\subsubsection{Other Languages}
 
\paragraph{XHTML}

This is a markup language used to design websites in similar fashion to regular
HTML but with stricter rules, and similar functionalities to XML, for example,
the use of namespaces. 

This project uses XHTML version 1.0 of it's Transitional
guidelines. Given it's more strict nature a common Web Browser will waste less
time processing web page content. 

Making a website that will be used by
thousands of students brings the concern of usability and accessibility. The
ErasmusLine project's Website will be developed under the guidelines of the Web
Content Accessibility Guidelines (version 2) and strive for a minimum first
level of acceptance. 

\paragraph{CSS}

This stylesheet language changes the presentation aesthetic of websites in a non
intrusive way, this way being able to switch style configuration rapidly without
touching the markup language files.

\subsubsection{Databases}

\paragraph{MySQL}

A relational database system designed for speed and rapidly building
content-oriented websites. This technology serves as the persistent data layer
in the application for various tasks such as the workflow of the Erasmus
Process, the users and students management, to the Data Warehousing used in the
P8-STATS package.

\subsubsection{Development Platforms}

\paragraph{SourceForge}

A project hosting website and development platform,
SourceForge\footnote{http://sourceforge.net/} provides infrastructures for the developers to quickly setup a project and use a varied number of features like
source code management systems, mailing list systems, forums, project management
applications, Wiki applications, etc.

\subsubsection{Development Tools}

\paragraph{Git}

A distributed source code management system used to keep track and monitor
source code versions, allowing the developer to determine release versions,
track code errors and share application code between other developers
distributed around the world.

This system saves the code in containers called repositories. Each time a
developer wishes to save a change made to the code, he commits file changes to
the local repository. A record of differences is then saved in the repository
along with a short description of the author. When the user is ready to share
their code changes with the rest of the team, he pushes those changes to the
central repository on the SourceForge site, so that other team members can pull
those changes into their local repositories.

Git was available to the project via SourceForge, which also provided an online
interface to check out repository activity and provided a RSS feed so that users
can be notified of recent changes to the code.

During the development of this project the team used the Git repository to keep
a record of the project planning, meetings reports, user manuals, the project
presentation and this project report, along with the applications code.

A usage example would consist of the following scenario (using a terminal
console):

\begin{enumerate}
  \item In a fresh installation, the developer would start by creating a copy of
  the SourceForge repository and subsequent directory structure and files, using the following command:
  \begin{verbatim}
  	git clone git://mutworange.git.sourceforge.net/gitroot/mutworange
  \end{verbatim}
  \item After making changes to the code, the developer would issue a commit:
  \begin{verbatim}
  	git commit -a -m “fixed a bug in the X package”
  \end{verbatim}
  Those changes would be saved locally.
  \item After making sure the code changes worked, the developer can share the
  changes with the team:
\begin{verbatim}
git push origin master
\end{verbatim}
Those changes would be merged with the central repository
  \item When new code changed are pushed to the central repository, the user can
  fetch them using the command:
\begin{verbatim}
git pull
\end{verbatim}
This fetches the latest changes to the code and merges them with your current
local work.
\end{enumerate}

Git also provides code branching functionality, which consists in creating an
alternate version or path in the application development. Branches are good for
creating new features on pre-existing code, but these features cannot be merged
yet with the “master” branch.

Another feature offered by Git is code tagging, enabling to checkout various
application versions of a branch in a certain context, for example a tag called
\emph{milestone-1} or \emph{release-1.2} would correspond to a point were code
development stopped and is ready to be deployed to the servers.

Source code management systems ensure that a team of developers keeps track of
their work in an effective and efficient way.

\paragraph{Eclipse / NetBeans}
Integrated Development Environments like NetBeans and Eclipse provide an
interface for code editing for various programming languages and has a set of
plugins to integrate other features in the interface, such as integrated Git
repository management without resorting to terminal commands.

\subsubsection{Other Tools}

\paragraph{Total Validator}

A cross-platform application designed to validate Web Sites
against established W3C standards. This application not only validates code for
XHTML but also validates Web Site accessibility guidelines, according to the
WCAG 2.0 specifications. Has a bare minimum, the Erasmusline application was
validated against WCAG 2.0 level A, but the application also validates on levels
AA and AAA where applied.

\paragraph{Delivery}

This application tries to make Javascript files smaller by removing comments,
spaces and new lines to provide faster downloads by the Web Browsers.

\paragraph{Google Docs}

During the course of this project some team members used Google Docs to write
technical reports that would later serve to compile into this final report.
Google Docs allowed the team members to create and edit text documents
concurrently and in real time with no fear of corrupting the documents. Google
Docs also supported versioning, which enabled us to see, for example, which team
member edited which part of the document.

\paragraph{Cacoo}

An online, realtime collaborative drawing tool which various team members can
use to produce UML diagrams, draw software deployment plans, etc. and was used
in several points of this project.

\paragraph{MySQL Workbench}
This tool helps the user design and project databases in a visual way by
creating Entity Relation models of the tables, specifying relational
constraints, create and edit indices, creating user roles and permissions,
default values for tables, etc. and later export those properties to a SQL file
to be immediately imported to a MySQL server.


\subsection{Conventions}

For the sake of usability this projects web-site and database content will be
using the UTF-8 character set encoding. This was a major concern since some
Erasmus partners use different alphabets – for example Greek and Turkish - and
the team had to use a   universal character set that accepted the various
alphabets existent in the European Continent. 

Other conventions were merely determined for development comprehension of the
team. While editing code the team member would have to follow a series of
guidelines to ensure that everyone could understand his code when reading it.
Some of these conventions were:

\begin{itemize}
  \item using UpperCamelCase on object class names 
  \item using lowerCamelCase on class methods
  \item documenting classes and methods using PHPDoc
  \item keep the code lines to a 80 character minimum, the excess code would be
  put in the following lines 
  \item opening brackets in the first line of the code statement
  \item everything else must have the default behavior enforced by the IDE, text
  editor, etc.
\end{itemize}

These conventions ensured the whole team understood each other and understood
what was being developed. While developing the Application Interface it was very important when designing the stylesheets not to interfere with the default behavior of the XHTML interface elements. It was agreed that default behavior would be left intact and set additional behavior by conventionally extending these said behaviors. This way it was ensured that when the team members developed in isolation, no necessary code rewrite would be needed.
