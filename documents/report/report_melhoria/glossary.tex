\chapter*{Glossary}
\begin{description}
\item[XHTML] eXtensible Hypertext Markup Language, a subset of the HTML markup
language which includes XML rules for more structured, accessible and faster
processing web-sites.
\item[MySQL] A popular, generally available Open Source Relational Database
Management System.
\item[UTF-8] A multibyte character encoding for Unicode. It has a wide range of
characters in which all European characters are included.
\item[Plonk] An open source PHP Library developed by Bramus Vandamme,
implementing the Model-View-Controller design architecture.
\item[MVC (Model View Controller)] A software architecture to separate the
logical code apart from the layout (view), the data management (model) and the
business Workflow (controller).
\item[cURL] A short version for “Client for URL”. Developed 1997 by Daniel
Stenberg and open source under the MIT – Licence.
\item[JSON (JavaScript Object Notation)] A text based standard design for data
interchange, used initially on Javascript code and spread to other technologies
due to it's lightweight syntax.
\item[3DES (Triple Data Encryption Standard)] The application of the Data
Encryption Standard algorithm three times on target data blocks, making it
harder to brute force hacking attempts. 
\item[HTTPS] The encrypted form of the Hypertext Transfer Protocol used on
websites.
\item[ODS] An operational data store is a database designed to integrate
data from multiple sources for additional operations on the data.
\item[DW] A Data Warehouse is a database used for statistical and reporting
purposes.
\item[OLAP (Online Analytical Processing)] Interactive analysis of data that
has been transformed from raw (operational) data into understandable 
enterprise-wide data.
\item[Objective] Tangent goals the EIS is trying to demonstrate and are later
translated to KPI's.
\item[KPI (Key Performance Indicator)] Statistical value that indicates
progress or predicts future progress of a business process.
\item[Dimension] A dimension is a structural attribute acting as an index for
identifying measures within a multidimensional data model. 
A dimension is basically a domain, which may be possibly partitioned into an 
hierarchy of levels. For example, in the context of selling goods, possible 
dimensions are product, time, and geography; chosen dimension levels may be 
Product category, Month, and District.
\item[Measure] A measure is a point into the multidimensional space. A measure
is identified if for each dimension a single value is selected. For example, 
a “sales volume” measure is identified by giving a specific product, 
a specific sale time, and a specific location.
\item[Drill-Down] The navigation among levels of data raging from higher level
summary (up) to lower level summary or detailed data (down). The drilling paths 
may be defined by the hierarchies within dimensions or other relationships that 
may be dynamic within or between dimensions. An example query is: for a 
particular product category, find detailed sales data for each office by date.
\item[Roll-Up] The querying for summarized data. Aggregation involves
computing the data relationships (according to attribute hierarchy within 
dimensions or to cross-dimensional formulas) for one or more dimensions. 
For example, sales offices can be rolled-up to districts and districts 
rolled-up to regions; the user may be interested in total sales or 
percent-to-total.
\item[Slice and Dice] The process employed by users to explore and query
multidimensional information within a OLAP cube interactively.
\item[ETL(Extract,Transform,Load)] Techniques used to integrate data from
heterogeneous data sources into another data store.
\item[Data Refreshment Plan] Rules and guidelines to integrate information
into the Data Warehouse.
\item[SSD] Solid State Drives are disk drives that can be accessed like a
conventional Hard Drive but instead of electro-magnetized metal plates, SSD’s 
use microchips to store information, like a conventional USB disk pen.
\item[btrfs] The B-Tree File System is a GPL licensed file system for GNU/Linux
Operating Systems. Among other features, it supports crash recovery, snapshots,
transactions and defragmentation.
\end{description}