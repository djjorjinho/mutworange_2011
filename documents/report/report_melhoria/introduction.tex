\chapter{Introduction} \label{ch:intro}

\section{Project Scope} \label{sec:pscope}
The MUTW (Multinational Undergraduate Team Work) Project is created by teachers
and students from all over the Europe. It's main purpose is to collaboratively
design a functional web application which is going to help the undergraduate
students and coordinators avoid the bureaucracy of filling all the necessary
forms and speed up the process, in case they want to use the Erasmus Program,
and also keep statistical records of what is happening during each student's
Erasmus mobility process.

The site which is called \emph{Erasmusline} is approachable by the students by
using their unique password given by their University and of course by the
coordinators who will work as administrators for the site. Furthermore, this
project has deeper goals to be achieved. Students have to work all together for
three months in order to make the site completely functional. This requires
teamwork and communication spirit. As a result, the participants will not only
practice their English language skills, but also their skills in planning,
analysis and technical expertise.

\section{Project Description}

The participants are students from eleven different Universities throughout
Europe which use the Erasmus Program. Each University offers two students for
the project. Every university has its own unique part in the project to
prepare, and during these three months, pieces are getting together in order to
make the final result, the site.

Eventually, this project turns out to be a unique and valuable experience which
can open new doors to the team's academic world. The objective of the project 
is to create a totally functional website which is going to be used by all
actors involved in the Erasmus program, that includes the students and
coordinators.

The students will be able to fill the forms needed for getting accepted by the
University abroad and the coordinators will be able to log in as administrators,
so they will be having access in every form in order to make the appropriate
moves for communicating with the other University.

Coordinators and students will also be able to consult global and institutional
statistics, generally for decision making - for example, students who seek to
study in institutions with specific traits - and detecting trends or anomalies
- in case of executive staff members who need to analyze the Erasmus process
success or failure in their own institution.

The beginning of the project took place in Kiel,Germany where all the
participants met for the first time and each team made its first view of the
project in a presentation. That presentation included the theoretical timeline
until the final presentation and what was the plan for creating the website step
by step.

There are two teams which take part in the project, the orange and the blue
team. Each team consists of six Universities from all over Europe and in the
final presentation the coordinators will evaluate each team according to the
work which was done these three months and the presentation itself.
Participants are gonna be also evaluated separately one by one.

\subsection{Project Planning}

\subsection{Project Meetings}

\section{Project Contributions}

\section{Report Structure}