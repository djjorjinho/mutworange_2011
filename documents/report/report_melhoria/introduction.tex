\chapter{Introduction} \label{ch:intro}

\section{Project Scope} \label{sec:pscope}
The MUTW (Multinational Undergraduate Team Work) Project is created by teachers
and students from all over the Europe. It's main purpose is to collaboratively
design a functional web application which is going to help the undergraduate
students and coordinators avoid the bureaucracy of filling all the necessary
forms and speed up the process, in case they want to use the Erasmus Program,
and also keep statistical records of what is happening during each student's
Erasmus mobility process.

The site which is called \emph{Erasmusline} is approachable by the students by
using their unique password given by their University and of course by the
coordinators who will work as administrators for the site. Furthermore, this
project has deeper goals to be achieved. Students have to work all together for
three months in order to make the site completely functional. This requires
teamwork and communication spirit. As a result, the participants will not only
practice their English language skills, but also their skills in planning,
analysis and technical expertise.

\section{Project Description}

The participants are students from eleven different Universities throughout
Europe which use the Erasmus Program. Each University offers two students for
the project. Every university has its own unique part in the project to
prepare, and during these three months, pieces are getting together in order to
make the final result, the site.

Eventually, this project turns out to be a unique and valuable experience which
can open new doors to the team's academic world. The objective of the project 
is to create a totally functional website which is going to be used by all
actors involved in the Erasmus program, that includes the students and
coordinators.

The students will be able to fill the forms needed for getting accepted by the
University abroad and the coordinators will be able to log in as administrators,
so they will be having access in every form in order to make the appropriate
moves for communicating with the other University.

Coordinators and students will also be able to consult global and institutional
statistics, generally for decision making - for example, students who seek to
study in institutions with specific traits - and detecting trends or anomalies
- in case of executive staff members who need to analyze the Erasmus process
success or failure in their own institution.

The beginning of the project took place in Kiel,Germany where all the
participants met for the first time and each team made its first view of the
project in a presentation. That presentation included the theoretical timeline
until the final presentation and what was the plan for creating the website step
by step.

There are two teams which take part in the project, the Orange and Blue
teams. Each team consists of six Universities from all over Europe and in the
final presentation the coordinators will evaluate each team according to the
work which was done these three months and the presentation itself.
Participants are gonna be also evaluated separately one by one.

This Project report will account for the process of planning and development of
the Orange team, but mainly for the Portuguese members and their assigned
project package, P8-STATS.

\subsection{Project Planning}

The Project development was controlled using the software tool
OpenProj\footnote{http://openproj.org} and editing the existing Gantt Chart,
accessible by the team's members in the SourceForge Git
repository\footnote{http://mutworange.git.sourceforge.net/}.

Every package member was responsible for filling out
their package Work Breakdown Structure, assign each deliverable time boxes and
regularly filling the percentage of their work done.

\fig{img/gantt_openproj.png}{The project's Gantt Diagram on
OpenProj}{img:gantt_openproj}{0.6}

The detailed Gantt Chart can be found in the chapter \emph{\nameref{apx5}}, page
\pageref{apx5} of this report.


Apart from checking the project's and each individual package's performance, the
team acted on democratic decision - when selecting the tools, when resolving
open ended questions – recurring to presential meetings through the software
tool Adobe Connect\footnote{http://www.adobe.com/products/adobeconnect.html},
IRC channels, direct email and the SourceForge Mailing List system.


Given the democratic nature of the team, each package team would also
self-manage their own package planning, using their own tools but always
reporting to a general Gantt Chart.

\ \newline
This project was developed during the second semester of 2011, between the
21$^{st}$ of March and the 17$^{th}$ of June. The following description lists
show each packages main tasks and milestones, each with their corresponding
time box - according to each country member's perspective.

\ \newline
\textbf{P1-CONFIG (Germany)} 
\begin{itemize}
\item Database \hfill 21/03 – 29/04 \begin{itemize}
  \item Design
  \item SQL Script
  \item Installation Script
\end{itemize}
\item Design \hfill 25/03 – 15/04 \begin{itemize}
  \item Design Mock Templates
  \item Integration with Plonk Templating library
\end{itemize}
\item User Account \hfill 18/04 – 29/04 \begin{itemize}
  \item Login functionality
  \item User roles and permissions
  \item Account page
\end{itemize}
\end{itemize}

\ \newline
\textbf{P2-INFOX (Germany)} 
\begin{itemize}
\item Discussion about communication Protocol \hfill 28/03 – 01/04
\item Define Security System \hfill 04/04 – 29/04
\item Programming Outgoing data	\hfill 02/05 – 13/05
\item Programming Incoming data	\hfill 02/05 – 13/05
\end{itemize}


\ \newline
\textbf{P3-ALERT (Bulgaria)} 
\begin{itemize}
\item Research	\hfill 22/03 – 06/04
\item Making the daemon	 \hfill 07/04 – 20/04
\item Synchronization with the others part of the project	\hfill 21/04 – 27/04
\item Email sending system	\hfill 28/04 – 05/04
\item Pop-ups	\hfill 28/04 – 05/04
\item GUI		\hfill 05/05 – 09/05
\item Testing	\hfill 10/05 – 23/05
\end{itemize}


\ \newline
\textbf{P4-OUT (Belgium)} 
\begin{itemize}
\item Collect Different forms	\hfill14/03 – 01/04
\item Setting up chart with the form flow	\hfill25/03 – 29/03
\item Agree with P-IN on layout forms		\hfill28/03 – 01/04
\item Determine which forms can be common		\hfill04/04 – 08/04
\item Implement non/common forms		\hfill11/04 – 29/04
\item Creating ECTS forms		\hfill28/03 – 01/04
\item Implement flow of forms		\hfill11/04 – 10/05
\item Integrate information  exchange module		\hfill18/04 – 22/04
\item Intensive testing of flow		\hfill10/05 – 31/05
\end{itemize}


\ \newline
\textbf{P5-IN (Greece)} 
\begin{itemize}
\item Research 
\item Develop the IN student forms
\item Testing
\item Report
\end{itemize}

\ \newline
\textbf{P6-EXAM (Bulgaria)} 
\begin{itemize}
\item Research
\item Making the student module
\item Making the home coordinator module
\item Making the host coordinator module
\item Synchronization with other parts of the project
\item Testing
\item Report
\end{itemize}

\ \newline
\textbf{P7-MATCH (Iceland)} 
\begin{itemize}
\item Research
\item Develop data scrapper to match dummy data
\item Develop production code
\item Integration tests
\item Testing
\item Report
\end{itemize}

\ \newline
\textbf{P8-STATS (Portugal)} 
\begin{itemize}
\item Analysis and Requirements gathering	\hfill14/03 – 29/03
	\begin{itemize}
	  \item Business Rules
	  \item Key Performance Indicators
	  	\begin{itemize}
	  		\item Measures
			\item Dimensions
		\end{itemize}
	   \item Report
	\end{itemize}
\item Data Warehouse		\hfill30/03 – 19/05
	\begin{itemize}
		\item Database schema
		\item Define data sources
		\item Architecture
		\begin{itemize}
		  \item DW ETL / Data refreshment
		  \item ODS and Metadata structure
		  \item Indexing Solutions
		\end{itemize}
		\item Implementation
		\begin{itemize}
		  \item DW ETL / Data refreshment
		  \item ODS and Metadata structure
		  \item OLAP module development
		\end{itemize}
		\item Deployment
		\begin{itemize}
		  \item DW ETL / Data refreshment
		  \item ODS and Metadata tables
		  \item Data Marts – Local and Remote DW
		\end{itemize}
		\item Report
	\end{itemize}
\item EIS Interface	\hfill19/05 – 06/06
	\begin{itemize}
	 	\item Architecture
		\item User Stories
		\item Implementation
		\item Deployment
		\item Report
	\end{itemize}
\end{itemize}

\ \newline
\textbf{ErasmusLine (Orange Team)} 
\begin{itemize}
 	\item Integration		\hfill16/05	– 31/05
 	\begin{itemize}
 	  \item Integration Tests
 	  \item Debugging
 	 \end{itemize}
	\item Final Report		\hfill01/06 – 17/06
	\begin{itemize}
	  \item User Manual
	\end{itemize}
	\item Presentation	\hfill08/06 – 17/06
	\begin{itemize}
	  \item Presentation Notes
	\end{itemize}
\end{itemize}

\subsection{Project Meetings}

As agreed on the first presential meetings, the team met initially remotely each
week on mondays on 18 o'clock according to the UTC-0 time zone. Each meeting
would last from fifty minutes to two hours if necessary. During the requirements
gathering and development stages of the project, the topics in each meeting
discussed by each package members, were the in the following list:

\begin{itemize}
  \item Report last week's activity 
	\item Plan next week's activity 
	\item Express difficulties and problems encountered, but not solve them right
away
\end{itemize}

Each package team would prepare in advance what to express in this first part of
the meeting to keep it short and quickly address issues in the third
topic. This first part of the meeting lasts twenty minutes and helps the team
stay updated and motivated with the evolution of the project. 

After each team reported these topics, the whole team would follow-up with
assigning team members to help solve the expressed problems in the third topic.
If any other meetings were needed to solve pending issues, the affected members
would arrange extra meetings during the week.

Initially, each package team would prepare and assume control of the meetings in
a rotational fashion each week, at least once. This proved to be non-productive
given that some team members didn't prepare nor had the skills to conduct the
meetings.

After a democratic vote from the team, it was decided unanimously that
the P8-STATS (Portugal) team should prepare and manage the meetings as well as
the Orange team.

The following sub-sections are a description of most of the meetings, including
the dates where held, as well the participants and discussed topics. All the members
attended these meetings using the Adobe Connect collaboration platform
(discussed on page \pageref{aconnect}), from their individual countries and homes.

\newpage

\subsubsection{Meeting 21/03/2011}
\normalsize
\begin{verbatim}
Date: 21/03/2011
Start Hour: 5:05pm UTC
End Hour: 6:30pm UTC

Attending: Arne Reimer, Pedro Ferreira, Daniel Lopes, Nathan Assche, Stephan
Polet, Aggeliki Katsiampouri, Mountrakis Stefanos, Thordur Bjornsson, Gudmundur
Hallgrimsson, Ina Ivanova, Zvezdomir Tsvyatkov

Topics:

* Status report (2-3min per country)
* Discussion and final decision of meetings and communication protocols
  (5-10min)
* Initial presentation of rules of defining future deadlines
* Suggestions and final remarks (5-10min)
* Reminder to set the task list until the next meeting (gantt chart)
* Team discussion about the problems previously presented (5-10min)
\end{verbatim}

\vspace{50px}
\subsubsection{Meeting 28/03/2011}
\normalsize
\begin{verbatim}
Date: 28/03/2011
Start Hour: 18:30 UTC
End Hour: 19:45 UTC

Attending: Arne Reimer, Arne Lipfert, Pedro Ferreira, Daniel Lopes,
Nathan Assche, Stephan Polet, Aggeliki Katsiampouri,
Thordur Bjornsson, Gudmundur Hallgrimsson, Ina Ivanova

Topics:
1. What has been done

* Germany: Gantt Chart, SQL-File, Forms, Templates
* Belgium: tutorial, register-page, collecting forms
* Portugal: business Workflow, statistical indicators, designing data model
* Greece: designed the needed forms, client validation, server validation
* Bulgaria: learning plonk and git, creating some more db-tables
* Iceland: did some test with php and python to integrate it later

2. Discussion
Germany will present up to Three Layouts until next meeting.

1. Infox
Germany will try to get some more Information about how to transfer Data.
Everybody agrres that it has to be done very quickly because Portugal/Belgium
and Bulgaria needs this infomation for their work.

2. Plonk
Belgium will try to get some more information about how to implement php-code
into a template.
\end{verbatim}

\vspace{50px}
\subsubsection{Meeting 04/04/2011}
\normalsize
\begin{verbatim}
Date: 04/04/2011
Start Hour: 17:30 UTC
End Hour: 19:45 UTC

Attending: Arne Reimer, Arne Lipfert, Pedro Ferreira, Daniel Lopes,
Nathan Assche, Stéphane Polet, Mountrakis Stefanos, Ina Ivanova

Topics:

* What have we done this week ( each team ) + problems/questions
	
* Database ( discussion )

* Main design: Germany presented 2 layouts

* Signatures 
	- Each team should ask their international office which forms really need 
	an official signature and which ones we can skip.

* Pre-candidate forms of all the teams ( and other .. ) + who provides
internship
    - We only received the pre-candidate form from Portugal ..
	- Need internship forms as quickly as possible, because this is an important
	part of our module.

* Information exchange
\end{verbatim}

\vspace{50px}
\subsubsection{Meeting 07/04/2011}
\normalsize
\begin{verbatim}
Date: 07/04/2011 
Start Hour: 09:00 UTC
End Hour: 10:00 UTC

Attending: Arne Reimer, Arne Lipfert, Pedro Ferreira, Daniel Lopes,
Nathan Assche, Stephan Polet

Topics:

* Discussion back and forth about the DB:
	a)Adding a Company table;
	b)The deployment apparently won't affect the DB design;
	c)Leaning towards not using a central repository (deployment wise).
		
\end{verbatim}

\vspace{50px}
\subsubsection{Meeting 25/04/2011}
\normalsize
\begin{verbatim}
Date: 25/04/2011
Start Hour: 18:00 UTC
End Hour: 18:50 UTC

Attending: Arne Reimer, Arne Lipfert, Pedro Ferreira, Daniel Lopes,
Nathan Assche,  Aggeliki Katsiampouri, Mountrakis Stefanos,

Topics:

1. What has been done
2. Plonk installation integration
3. Assign next week's meeting coordinators
\end{verbatim}

\vspace{50px}
\subsubsection{Meeting 09/05/2011}
\normalsize
\begin{verbatim}
Date: 09/05/2011
Start Hour: 18:00 UTC
End Hour: 20:00 UTC


Attending: Arne Reimer, Arne Lipfert, Pedro Ferreira, Daniel Lopes,
Nathan Assche, Stéphane Polet, Aggeliki Katsiampouri, Zvezdomir Tsvyatkov

Topics:

* What have we done this week ( each team )  problems/questions
    
* Zvezdomir had some problems getting the site online and with the login
 functionality. We will try to figure out the problem this week.

* Belgium and Germany are going to put 2 applications online in order to start
 testing the infox module.

* Portugal will coordinate the next meetings. ( vote )

* Greece will finish their forms by wednesday and then push them to GIT.
\end{verbatim}

\vspace{50px}
\subsubsection{Meeting 16/05/2011}
\normalsize
\begin{verbatim}
Date: 16/05/2011
Start Hour: 18:05 UTC
End Hour: 19:00 UTC

Attending: Pedro Ferreira, Daniel Lopes, Arne Reimer,
Nathan Assche, Stéphane Polet, Ina Ivanova, Stefanos Mountrakis

Topics:

- What did the team do this past week
- What does the team plan to do next
- Any doubts/issues about their package
- Percentage of work done
- Creation of campus servers to test application workflow and communication
- DB, Infox changes succsessful, needs further testing
- Next meeting to discuss integration issues this week
\end{verbatim}

\vspace{50px}
\subsubsection{Meeting 23/05/2011}
\normalsize
\begin{verbatim}
Date: 23/05/2011
Start Hour: 18:05 UTC
End Hour: 19:00 UTC

Attending: Pedro Ferreira, Daniel Lopes, Arne Reimer,
Nathan Assche, Stéphane Polet, Stefanos Mountrakis, Arne Lipfert, Gudmundur Hallgrimsson

Topics:

- What did the team do this past week
- What does the team plan to do next
- Any doubts/issues about their package
- Is the integration going along well?
- How can we test the application?
- Given the time we have left, do we have to sacrifice features?
\end{verbatim}

\vspace{50px}
\subsubsection{Meeting 30/05/2011}
\normalsize
\begin{verbatim}
Date: 30/05/2011
Start Hour: 18:05 GMT
End Hour: 19:00 GMT

Attending: Pedro Ferreira, Daniel Lopes, Arne Reimer, Arne Lipfert, Nathan
Assche, Stephane Polet, Stefanos Mountrakis, Gudmundur Hallgrimsson, Zvezdomir
Tsvyatkov, Aggeliki Katsiampouri, Ina Ivanova

Topics:
- What did the team do this past week
- What is necessary to do this week:
     1. Finish code development
     2. Tie up loose ends
     3. Gather team members and do a live demo of the running application 
     	and fix whatever is left
\end{verbatim}

\newpage

\section{Project Contribution}

As a whole, the ErasmusLine project delivers a system that will:

\begin{itemize}
  \item Organize and automate the student's and staff member's Erasmus workflow
  \item Provide a central point of communication between Higher Education
  Institutions
  \item Organizes information in a single web application
  \item Removes heterogeneous means of communication
  \item Tracks efficiency and efficacy patterns for each Institution
  \item Provides decision support tools for the Institutions' executive members
\end{itemize}

The inovation of this project is that it pushes the students into an
``Enterprise'' scenario that not only will test their knowledge but also make
them deal with dificult decisions that will take the students out of their
confort zone and possibly making the wrong answer, but notheless taking the
initiative and sustaining the decisions with strong arguments.


In the Portuguese team's case, it will allow to put in practice the experience
gained during their learning period of the License degree, but also explore
and gain new knowledge on Decision Support systems, something that
currently in only lectured in a Master's degree.


\section{Report Structure}


