\chapter{Introduction} \label{ch:intro}

\section{Project Scope} \label{sec:pscope}
The MUTW (Multinational Undergraduate Team Work) Project is created by teachers
and students from all over the Europe. It's main purpose is to collaboratively
design a functional web application which is going to help the undergraduate
students and coordinators avoid the bureaucracy of filling all the necessary
forms and speed up the process, in case they want to use the Erasmus Program,
and also keep statistical records of what is happening during each student's
Erasmus mobility process.

The site which is called \emph{Erasmusline} is approachable by the students by
using their unique password given by their University and of course by the
coordinators who will work as administrators for the site. Furthermore, this
project has deeper goals to be achieved. Students have to work all together for
three months in order to make the site completely functional. This requires
teamwork and communication spirit. As a result, the participants will not only
practice their English language skills, but also their skills in planning,
analysis and technical expertise.

\section{Project Description}

The participants are students from eleven different Universities throughout
Europe which use the Erasmus Program. Each University offers two students for
the project. Every university has its own unique part in the project to
prepare, and during these three months, pieces are getting together in order to
make the final result, the site.

Eventually, this project turns out to be a unique and valuable experience which
can open new doors to the team's academic world. The objective of the project 
is to create a totally functional website which is going to be used by all
actors involved in the Erasmus program, that includes the students and
coordinators.

The students will be able to fill the forms needed for getting accepted by the
University abroad and the coordinators will be able to log in as administrators,
so they will be having access in every form in order to make the appropriate
moves for communicating with the other University.

Coordinators and students will also be able to consult global and institutional
statistics, generally for decision making - for example, students who seek to
study in institutions with specific traits - and detecting trends or anomalies
- in case of executive staff members who need to analyze the Erasmus process
success or failure in their own institution.

The beginning of the project took place in Kiel,Germany where all the
participants met for the first time and each team made its first view of the
project in a presentation. That presentation included the theoretical timeline
until the final presentation and what was the plan for creating the website step
by step.

There are two teams which take part in the project, the Orange and Blue
teams. Each team consists of six Universities from all over Europe and in the
final presentation the coordinators will evaluate each team according to the
work which was done these three months and the presentation itself.
Participants are gonna be also evaluated separately one by one.

This Project report will account for the process of planning and development of
the Orange team, but mainly for the Portuguese members and their assigned
project package, P8-STATS.

\subsection{Project Planning}

The Project development was controlled using the software tool
OpenProj\footnote{http://openproj.org} and editing the existing Gantt Chart,
accessible by the team's members in the SourceForge Git
repository\footnote{http://mutworange.git.sourceforge.net/}.

Every package member was responsible for filling out
their package Work Breakdown Structure, assign each deliverable time boxes and
regularly filling the percentage of their work done.

\fig{img/gantt_openproj.png}{The project's Gantt Diagram on
OpenProj}{img:gantt_openproj}{0.6}

The detailed Gantt Chart can be found in the chapter \emph{\nameref{apx5}}, page
\pageref{apx5} of this report.


Apart from checking the project's and each individual package's performance, the
team acted on democratic decision - when selecting the tools, when resolving
open ended questions – recurring to presential meetings through the software
tool Adobe Connect\footnote{http://www.adobe.com/products/adobeconnect.html},
IRC channels, direct email and the SourceForge Mailing List system.


Given the democratic nature of the team, each package team would also
self-manage their own package planning, using their own tools but always
reporting to a general Gantt Chart.

\ \newline
This project was developed during the second semester of 2011, between the
21$^{st}$ of March and the 17$^{th}$ of June. The following description lists
show each packages main tasks and milestones, each with their corresponding
time box - according to each country member's perspective.

\ \newline
\textbf{P1-CONFIG (Germany)} 
\begin{itemize}
\item Database \hfill 21/03 – 29/04 \begin{itemize}
  \item Design
  \item SQL Script
  \item Installation Script
\end{itemize}
\item Design \hfill 25/03 – 15/04 \begin{itemize}
  \item Design Mock Templates
  \item Integration with Plonk Templating library
\end{itemize}
\item User Account \hfill 18/04 – 29/04 \begin{itemize}
  \item Login functionality
  \item User roles and permissions
  \item Account page
\end{itemize}
\end{itemize}

\ \newline
\textbf{P2-INFOX (Germany)} 
\begin{itemize}
\item Discussion about communication Protocol \hfill 28/03 – 01/04
\item Define Security System \hfill 04/04 – 29/04
\item Programming Outgoing data	\hfill 02/05 – 13/05
\item Programming Incoming data	\hfill 02/05 – 13/05
\end{itemize}


\ \newline
\textbf{P3-ALERT (Bulgaria)} 
\begin{itemize}
\item Research	\hfill 22/03 – 06/04
\item Making the daemon	 \hfill 07/04 – 20/04
\item Synchronization with the others part of the project	\hfill 21/04 – 27/04
\item Email sending system	\hfill 28/04 – 05/04
\item Pop-ups	\hfill 28/04 – 05/04
\item GUI		\hfill 05/05 – 09/05
\item Testing	\hfill 10/05 – 23/05
\end{itemize}


\ \newline
\textbf{P4-OUT (Belgium)} 
\begin{itemize}
\item Collect Different forms	\hfill14/03 – 01/04
\item Setting up chart with the form flow	\hfill25/03 – 29/03
\item Agree with P-IN on layout forms		\hfill28/03 – 01/04
\item Determine which forms can be common		\hfill04/04 – 08/04
\item Implement non/common forms		\hfill11/04 – 29/04
\item Creating ECTS forms		\hfill28/03 – 01/04
\item Implement flow of forms		\hfill11/04 – 10/05
\item Integrate information  exchange module		\hfill18/04 – 22/04
\item Intensive testing of flow		\hfill10/05 – 31/05
\end{itemize}


\ \newline
\textbf{P5-IN (Greece)} 
\begin{itemize}
\item Research 
\item Develop the IN student forms
\item Testing
\item Report
\end{itemize}

\ \newline
\textbf{P6-EXAM (Bulgaria)} 
\begin{itemize}
\item Research
\item Making the student module
\item Making the home coordinator module
\item Making the host coordinator module
\item Synchronization with other parts of the project
\item Testing
\item Report
\end{itemize}

\ \newline
\textbf{P7-MATCH (Iceland)} 
\begin{itemize}
\item Research
\item Develop data scrapper to match dummy data
\item Develop production code
\item Integration tests
\item Testing
\item Report
\end{itemize}

\subsection{Project Meetings}

\section{Project Contributions}

\section{Report Structure}