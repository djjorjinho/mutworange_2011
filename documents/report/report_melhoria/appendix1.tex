\chapter{Appendix 1: Early Database Deployment Draft}
Written by P8-STATS team.
\newline \newline
Decentralized Idea:\\
The application is self autonomous and each institution is responsible to
install and host it. (Example: brain cells implementation?)\\

\noindent What does this imply?

\noindent Every Institution will have their own webpage (that at the beginning
it’s the one designed by us but we could give the incentive to each one make
their own interface) and Database.\\
This way the login problems of the accounts not being centralized are
nullified.\\
How it will work:

\fig{img/apx1_deploy.png}{Application Deployment
throughout Institutions}{img:apx1_deploy}{0.7}

\textbf{A} - Given this implementation, information between Institutions has to
be shared and we have to decide how we will accomplish this.\\

\textbf{A1} -
\begin{itemize}
\item[a)] We can use a repository that controls and keeps track of all Institutions and
the new ones, therefore the application when installed accesses this repository
to ‘register itself’ and get the list of all the running ones, after that, an
alert is generated to all the actives ones so they can update themselves with
the “new born”.
\item[b)] Taking into consideration that Institution will need to become partners first
before exchanging students, if there’s no other reason for a central repository
we can avoid it and say that they have to agree on partnership first by their
own communications and if accepted they add each other on the application.
\end{itemize}

\textbf{A2} - Database mirroring is completely out of question given the huge
amount of Databases that can exist so the information will be separated,
therefore we will have o create some kind of communication protocol between Institutions for when
they need to get info that’s stored on the other side and so on.\\

Pros:
\begin{itemize}
\item Account management and ‘power attribution’ simplified since every Institution
is responsible for them self -Law problems with info also probably solved
\item Pre-candidate forms being different can also be solved by this (everyone
is responsible for their own)
\end{itemize}

Cons:
\begin{itemize}
\item Information about the host coordinator, the student that comes from far away,
stuff like that, needs to be accessed
\item Database design might be altered because some extra info from receiving
students might be stored and stuff like that depending on the implementation,
infox will have to support all the communications about this info 
\item Harder to update/patch
\end{itemize}

Problems:
\begin{itemize}
  \item The info that isn't stored and needs to be exchanged, how to work around
  that? Are there any law implications on how they are shared? Can they be temporarily
stored or they can just be accessed to ‘read only’? 
\item Some kind of central control will be required if the central repository is to
be implemented (extra staff/administrators needed).
\end{itemize}

Possible Solutions:
\begin{itemize}
  \item After a process is accepted, a ticket is created between the receiving and
giving Institution, so when they need info about each other they can use that
ticket that will block their view only to what’s allowed and as soon the out
process passes, the ticket expires.
\item Going with A1 b avoids that, probably the best way to go but still with some
implications.
\end{itemize}
