\chapter{Technical Description} \label{ch:technical}

The P8-STATS package, the integral part of the statistical core developed by the
Portuguese team consists in developing an Executive Infortmation System,
integrated in the ErasmusLine ecosystem.

Executive Information Systems (EIS) are Information Systems that access large
amounts of business related information from various sources - internal and
external to the organization - and provides decision support tools to the
organization's senior executives.

These tools allow the executive managers to highlight patterns in the
business process by analyzing, comparing and determining trends in the provided
mediums - spreadsheets, graphic charts, pivot tables, reports, etc.

As an example of EIS use, consider the following scenario:

\begin{itemize}
\item[] Erasmus executive staff and coordinators can do a periodical check on
student enrollment information, i.e. what countries do they come from and what courses
are their main choices. Using a web browser, executive memebers can access that
information in a personalized fashion by presenting subsets of the information -
views. These views can be "drilled-down" to minute levels of information or
"rolled-up" to display a broader view.
\end{itemize}

Since executive users are not necessarily data analysts, the EIS user interface
must be as simple as possible, but without sacrificing presentation dynamics.

With these criteria in mind, the following key requirements for an EIS where
identified\cite{eis:psu}:

\begin{itemize}
  \item Cross Platform
  \item Ease of Use
  \item Limited Training
  \item Quick Response
  \item Process Large Volumes of Data
  \item Deployment Through the Web
  \item Easy Graphical Presentation Options
  \item Ability to Access Subsets of Data (Drill Down)
\end{itemize}

\section{Architecture}
An EIS can be divided into three main levels of architecture\cite{eis:dlbi}, that
covers the technology used to extract, analyze and visualize information from other data
sources with a friendly and flexible user interface. The levels discussed in the
next sections are the following:

\begin{description}
  \item[Data Management] Data gathering and transformation, it's lifecycle and
  target storage.
  \item[Model Management] How the data is transformed and retrieved for
  analytical purposes.
  \item[Data Visualization] Which visual tools are used to represent the
  underlying system architecture.
\end{description}


\section{Data Management}

The Data Management Level deals with the extraction, analysis and processing of
the internal and external data sources and passes that data through an ETL
process - Extract, Transform and Load - that organizes and aggregates the data
into the Data Warehouse. The use of ETL and Data Warehouses can be more
efficient to the EIS due to it's data treatment and Warehouse database model
schemata.

In this project's context, the sole Data Source for the Data Management level is
the main Erasmusline database that provides information for 
\begin{itemize}
	\item The Students
	\item The Higher Education Institutions
	\item The lectured Courses
	\item Student Forms
	\item Exams
\end{itemize}

It is this data that will pass through the ETL stage.

\subsection{ETL process}

The ETL process parses the data and performs these possible
operations\cite{wiki:etl}:
\begin{itemize}
  \item Translation into database compatible rows / columns
  \item Translate code values
  \item Remove / Add / Translate code values
  \item Change data encoding
  \item Unify string value variations - i.e. 'Mr','Mister'; '1','One'; etc.
  \item Sorting values
  \item Data validation
  \item Summarize data
  \item Join data from various sources
  \item Generating surrogate keys
\end{itemize}

Also, ETL processes can make use of temporary database tables for intermediate
processing, called Operational Data Stores (ODS). These tables hold historical
data that can me mined for statistical information to be included in the Data
Warehouse.

For the P8-STATS package, the ETL process will be executed either by an event
trigger - saving data in the ODS’s when a certain workflow phase is reached -
and periodically - part of the data refreshment process to aggregate new
information to the Data Warehouse.

\subsubsection{Data Refreshment}
This phase of the ETL process is comprised of design choices, like data
structures, techniques to update and optimize the flow of information from the
data sources to the Data Warehouse and update cycle.

Based on the statistical overview document of the Erasmus Program of 2010 – in
the Annexes – it was verified that with the large volume of students mobility,
having a completely centralized database wouldn't be viable in terms of database
synchronization and efficiency - 198523 students got mobilized by 2747
Institutions in the academic year 2008/2009.

The process begins with the loading of information from the Operational Data
Stores.

\paragraph{Loading}
The information sent to the ETL component will be fetched from the application
data sources while other subsets of information are already stored in the ODS’s
or Metadata tables - these hold information previously gathered by the business
requisites - which is going to be processed by the ETL component and integrated
into the Data Warehouse.

To aggregate and integrate data to be viewed in the EIS, an update cycle for the
Data Warehouse must be determined. The information models that need to be
gathered over time are noted in the following table:  

\tab{img/refresh_cycle.png}{EIS update cycle for the Data Refreshment
phase}{tab:refresh_cycle}{0.7}

After the Process Efficiency data is integrated into the Institution’s local
Data Mart - a local database, part of the Data Warehouse - the Efficacy process
gathers the data from the various Institution’s Data Marts and additional data
to aggregate and integrate new data into the central Data Mart.

More on this discussion in the Deployment section, page
\pageref{sec:deployment} of this report.

\subsection{Data Warehouse}

\subsubsection{MySQL and Merge Tables}


