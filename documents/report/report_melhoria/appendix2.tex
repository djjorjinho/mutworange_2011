\chapter{Appendix 2: Early Email Example to Belgium}
Written by P8-STATS team.
\begin{verbatim}
Heya,

We have a site with all or almost all the info for ERASMUS apparently but
it's all in portuguese ( http://www.ipp.pt/index.php?a=25&id=104&sub=250
).

We can translate the forms if you like, but the idea of reaching an
uniformization seems unlikely(up to your investigation). In that case we
suggest looking into this idea/concept:

-Creating each and every form and storing them in file format( ala pdf, etc )

How?

-Given their location, presenting the correct form in lets say pdf and
they have to print, fill, sign, scan it and upload it back(ie just one
example, web forms seem impractical because you have to sign)

Why?

-The idea is to keep this expandable, we produce a product that, let's
say, is ready to support 5 universities but the design makes it able to
expand even while already deployed they just need to add the new pdf to
the page
giving us the responsibility of making the core encapsulated and well
documented in order for further improvements that really should be their
problem(ala adding the respective universities forms).


I would also like to ask and therefore suggest if your making this
planning with the Greeks since i believe you should both use the same
methods etc.


As a last reminder don't forget communication, ask ask ask ask ask and ask
any doubts or suggestions (ala the meetings post has 0 comments!!).


Any further doubt feel free to mail or add me in google talk -
ped.j.ferreira@gmail.com

Regards,

Pedro Ferreira
ISEP, Portugal
\end{verbatim}
